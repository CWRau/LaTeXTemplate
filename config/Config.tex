%!TEX root = ../Thesis.tex
%% Basierend auf TeXnicCenter-Vorlage von Mark Müller
%%                                        Willi Nüßer
%%                                        Waldemar Penner
%%                                        Ulrich Reus
%%                                        Frank Plass
%%                                        Oliver Tribeß
%%                                        Daniel Hintze
%%%%%

% Wählen Sie die Optionen aus, indem Sie % vor der Option entfernen
% Dokumentation des KOMA-Script-Packets: scrguide

%%%%%
%% Optionen zum Layout des Artikels %%
%%%%%
\documentclass[     %
	paper=A4,           % alle weiteren Papierformat einstellbar
	fontsize=12pt,      % Schriftgröße (12pt, 11pt (Standard))
	BCOR=12mm,          % Bindekorrektur, bspw. 1 cm
	DIV=14,             % breiter Satzspiegel
	parskip=half*,      % Absatzformatierung s. scrguide 3.1
	headsepline,        % Trennline zum Seitenkopf
	%footsepline,       % Trennline zum Seitenfuß
	%normalheadings,    % Überschriften etwas kleiner (smallheadings)
	listof=totoc,       % Tabellen & Abbildungsverzeichnis ins Inhaltsverzeichnis
	%bibtotoc,          % Literaturverzeichnis im Inhalt
	%draft              % Überlangen Zeilen in Ausgabe gekennzeichnet
	footinclude=false,  % Fußzeile in die Satzspiegelberechnung einbeziehen
	headinclude=true,   % Kopfzeile in die Satzspiegelberechnung einbeziehen
	final,              % draft beschleunigt die Kompilierung
	ngerman
]
{scrartcl}

% Links im PDF
\usepackage{hyperref}

\pdfinfo{ /ModDate ()  /CreationDate () }
\hypersetup{pdfinfo={ Creator={}, Producer={} }}

\pdftrailerid{}
\pdfsuppressptexinfo15


% Umlaute können verwendet werden
\usepackage[utf8]{inputenc}

\usepackage{xcolor}

\usepackage{etoolbox}

\usepackage{fixfoot}

\usepackage{rotating}

\usepackage{tikz}
\usetikzlibrary{
	arrows,
	positioning,
	calc,
	intersections,
	decorations.markings,
	fit,
	chains,
}

\tikzset{
	base/.style={
			draw,
			rectangle,
			align=center
		}
}

\usepackage{spreadtab}

\newtoggle{draft}\togglefalse{draft}
\toggletrue{draft}

\newcommand{\see}[2][sec]{\hyperref[#1:#2]{siehe \autoref*{#1:#2}}}
\newcommand{\seef}[1]{\see[fig]{#1}}
\newcommand{\seel}[1]{\see[lst]{#1}}
\newcommand{\seei}[1]{\see[itm]{#1}}

\newcommand{\seeb}[2][sec]{\hyperref[#1:#2]{(siehe \autoref*{#1:#2})}}
\newcommand{\seelb}[1]{\seeb[lst]{#1}}
\newcommand{\seefb}[1]{\seeb[fig]{#1}}
\newcommand{\seeib}[1]{\seeb[itm]{#1}}

\newcommand{\seep}[3][sec]{\hyperref[#1:#2]{siehe #3 in \autoref*{#1:#2}}}
\newcommand{\seepf}[2]{\seep[fig]{#1}{#2}}
\newcommand{\seepl}[2]{\seep[lst]{#1}{#2}}
\newcommand{\seepi}[2]{\seep[itm]{#1}{#2}}

\newcommand{\seepb}[3][sec]{\hyperref[#1:#2]{(siehe #3 in \autoref*{#1:#2})}}
\newcommand{\seepbf}[2]{\seepb[fig]{#1}{#2}}
\newcommand{\seepbl}[2]{\seepb[lst]{#1}{#2}}
\newcommand{\seepbi}[2]{\seepb[itm]{#1}{#2}}

\newcommand{\seea}[1]{\hyperref[app:#1]{siehe Anhang~\ref*{app:#1}}}
\newcommand{\seeab}[1]{\hyperref[app:#1]{(siehe Anhang~\ref*{app:#1})}}
\newcommand{\seepa}[2]{\hyperref[app:#1]{siehe #2 in Anhang~\ref*{app:#1}}}
\newcommand{\seepba}[2]{\hyperref[app:#1]{(siehe #2 in Anhang~\ref*{app:#1})}}

\newcommand{\seelz}[2]{\hyperref[lst:#1]{siehe \autoref*{lst:#1}, Zeile #2}}
\newcommand{\seelbz}[2]{\hyperref[lst:#1]{(siehe \autoref*{lst:#1}, Zeile #2)}}

\newcommand{\seeplz}[3]{\hyperref[lst:#1]{siehe #3 in \autoref*{lst:#1}, Zeile #2}}
\newcommand{\seeplbz}[3]{\hyperref[lst:#1]{(siehe #3 in \autoref*{lst:#1}, Zeile #2)}}


\newcommand{\ar}[2][sec]{\autoref{#1:#2}}
\newcommand{\arl}[1]{\ar[lst]{#1}}
\newcommand{\arf}[1]{\ar[fig]{#1}}
\newcommand{\ari}[1]{\ar[itm]{#1}}

\newcommand{\arb}[2][sec]{(\autoref{#1:#2})}
\newcommand{\arfb}[1]{\arb[fig]{#1}}
\newcommand{\arlb}[1]{\arb[lst]{#1}}
\newcommand{\arib}[1]{\arb[itm]{#1}}

\newcommand{\arlz}[2]{\hyperref[lst:#1]{\autoref*{lst:#1}, Zeile #2}}
\newcommand{\arlbz}[2]{\hyperref[lst:#1]{(\autoref*{lst:#1}, Zeile #2)}}


\newcommand{\ctapp}[2]{\footnote{\hyperref[app:#1]{#2}}}

\newcommand{\cit}[3]{\footcite[#2][#3]{#1}}
\newcommand{\ct}[2][]{\cit{#2}{#1}{}}
\newcommand{\vg}[1]{\ct[vgl.]{#1}}

\newcommand{\ctsep}{\textsuperscript{,}}

\usepackage{float}
\usepackage{ltablex}
%% \usepackage{fixltx2e}

%\setuptoc{toc}{totoc} % Inhaltsverzeichnis ins Inhaltsverzeichnis

% Neue Deutsche Rechtschreibung und Deutsche Standardtexte
\usepackage[ngerman]{babel}

\hyphenation{ein-zu-chec-ken}

% Echte Umlaute
\usepackage[T1]{fontenc}

% Latin Modern Font, Type1-Schriftart für nicht-englische Texte
\usepackage{lmodern}

% 1/2-zeiliger Zeilenabstand
\usepackage[onehalfspacing]{setspace}

% Für die Defenition eigener Kopf- und Fußzeilen
\usepackage{fancyhdr}

% Für die Verwendung von Grafiken
\usepackage[]{graphicx}
\graphicspath{{img/}}

% Bessere Tabellen
\usepackage{tabularx}

% Für die Befehle \toprule, \midrule und \bottomrule, z.B. in Tabellen
\usepackage{booktabs}

% Erlaubt die Benutzung von Farben
\usepackage{color}

% Verbessertes URL-Handling mit \url{http://...}
\usepackage{url}

% Listen ohne Abstände \begin{compactlist}...\end{compactlist}
\usepackage{paralist}

\iftoggle{draft} {
	% Ausgabe der aktuellen Uhrzeit für die Draft-Versionen
	\usepackage{datetime2}
	\DTMsetstyle{iso}
}

% Deutsche Anführungszeichen
\usepackage[babel,german=quotes]{csquotes}

\usepackage{amsmath}

% Verbessert das Referenzieren von Kapiteln, Abbildungen etc.
\usepackage[ngerman,capitalise]{cleveref}

% Konfiguration der Abbildungs- und Tabellenbezeichnungen
\usepackage[format=hang, font={footnotesize, sf}, labelfont=bf, justification=raggedright,singlelinecheck=false]{caption}

% Verbessert die Lesbarkeit durch Mikrotypografie
\usepackage[activate={true,nocompatibility},final,tracking=true,kerning=true,spacing=true,factor=1100,stretch=10,shrink=10]{microtype}

% Zitate und Quellenverzeichnis
\usepackage[
	style=authoryear-icomp,  % Zitierstil
	giveninits=false,        % false = Vornamen werden ausgeschrieben
	natbib=true,
	urldate=iso,            % "besucht am" - Datum
	%url=false,
	date=iso,
	seconds=true,
	dashed=false,
	maxcitenames=3,          % max. Anzahl Autorennamen in Zitaten
	maxbibnames=99,          % max. Anzahl Autorennamen im Quellenverzeichnis
	%backend=bibtex          % Ggf. für ältere Distributionen bibtex verwenden
	backend=biber
]{biblatex}
\iftoggle{draft} {
	\usepackage[printwatermark]{xwatermark}
}

% Bibliography
\bibliography{library/library}

% Keine Einrückung bei einem neuen Absatz
\parindent 0pt

% Ebenentiefe der Nummerierung
\setcounter{secnumdepth}{3}

% Gliederungstiefe im Inhaltsverzeichnis
\setcounter{tocdepth}{3}

% Tabellen- und Abbildungsverzeichnis mit Bezeichnung:
\usepackage[titles]{tocloft}

% Sourcecode-Listings
\usepackage{listings}

\definecolor{groovyblue}{HTML}{0000A0}
\definecolor{groovygreen}{HTML}{008000}
\definecolor{darkgray}{rgb}{.4,.4,.4}

\lstdefinelanguage{base}[]{Java}{
	keywordstyle=\color{groovyblue}\bfseries,
	stringstyle=\color{groovygreen}\ttfamily
}

\lstdefinelanguage{Powershell}[]{base}{
	keywords=[3]{CmdletBinding, param, string, function, ValidateSet, Switch, @, parameter, Mandatory, \$true, \$false, \$null, Microsoft, Win32, Registry, System, Security, Principal, NTAccount, SecurityIdentifier, IO, Path, Get, Command, Name, ErrorAction, SilentlyContinue, Invoke, WebRequest, UseBasicParsing, Expression, object, Content, foreach, like, StringSplitOptions, notlike, IsNullOrEmpty, ForegroundColor, Write, Host, Push, Location, Set, Encoding, Force, Split, Pop, NetFirewallRule, DisplayName, Group, Action, Direction, Enabled, Protocol, LocalPort, Measure, Object, eq, ne, Program, New, Value, PropertyType, Out, Null, ItemProperty, hive, key, Service, StartupType, Stop, Enable, PSRemoting, Start, AccessControl, AccessControlSections, RegistryKeyPermissionCheck, RegistryRights, RegistryAccessRule},
	morekeywords={in}
}

\lstdefinelanguage{Jenkinsfile}[]{base}{
	keywords=[3]{node, dir, fileExists, sh, script, returnStdout, properties, parameters, choice, choices, name, defaultValue, description, string, isUnix, error, withEnv, withCredentials, credentialsId, usernameVariable, passwordVariable, println, parallel, stash, includes, archiveArtifacts, artifacts, defaultExcludes, fingerprint, onlyIfSuccessful, step, \$class, projectName, filter, unstash, input, message, stage, build, job, value, milestone, withMaven, maven, globalMavenSettingsConfig},
	morekeywords={def, as, in, use, @NonCPS, groovy}
}

\lstset{
	breaklines=true,
	postbreak=\raisebox{0ex}[0ex][0ex]{\ensuremath{\color{red}\hookrightarrow\space}},
	numbers=left,
	inputencoding=utf8,
	emptylines=1,
	showlines=true
}

\lstloadlanguages{base, Powershell, Jenkinsfile}

% Bestimmte Warnungen unterdrücken
% siehe http://tex.stackexchange.com/questions/51867/koma-warning-about-toc
\usepackage{scrhack}

%% http://tex.stackexchange.com/questions/126839/how-to-add-a-colon-after-listing-label
\makeatletter
\begingroup\let\newcounter\@gobble\let\setcounter\@gobbletwo
\globaldefs\@ne \let\c@loldepth\@ne
\newlistof{listings}{lol}{\lstlistlistingname}
\endgroup
\let\l@lstlisting\l@listings
\makeatother

\renewcommand*\cftfigpresnum{Abbildung~}
\renewcommand*\cfttabpresnum{Tabelle~}
\renewcommand*\cftlistingspresnum{Listing~}
\renewcommand{\cftfigaftersnum}{:}
\renewcommand{\cfttabaftersnum}{:}
\renewcommand{\cftlistingsaftersnum}{:}
\settowidth{\cftfignumwidth}{\cftfigpresnum 99~\cftfigaftersnum}
\settowidth{\cfttabnumwidth}{\cfttabpresnum 99~\cftfigaftersnum}
\settowidth{\cftlistingsnumwidth}{\cftlistingspresnum 99~\cftfigaftersnum}
\setlength{\cfttabindent}{1.5em}
\setlength{\cftfigindent}{1.5em}
\setlength{\cftlistingsindent}{1.5em}

\renewcommand\lstlistlistingname{Listingverzeichnis}

% Style für Kopf- und Fußzeilenfelder
\pagestyle{fancy}
\fancyhf{}
\fancyhead[R]{\leftmark}
\fancyfoot[R]{\thepage}
\renewcommand{\sectionmark}[1]{\markboth{#1}{#1}}
\fancypagestyle{plain}{}

% Macro für Quellenangaben unter Abbildungen und Tabellen
\newcommand{\source}[1]{{\vspace{-1mm}\\\footnotesize\textsf{\textbf{Quelle:}} \textsf{#1}\par}}

% Anpassungen der Formatierung an Eclipse-Aussehen
% http://jevopi.blogspot.de/2010/03/nicely-formatted-listings-in-latex-with.html
%\definecolor{sh_comment}{rgb}{0.12, 0.38, 0.18 } %adjusted, in Eclipse: {0.25, 0.42, 0.30 } = #3F6A4D
%\definecolor{sh_keyword}{rgb}{0.37, 0.08, 0.25} % #5F1441
%\definecolor{sh_string}{rgb}{0.06, 0.10, 0.98} % #101AF9
% Für Druckausgabe sollte alles schwarz sein
\definecolor{sh_comment}{rgb}{0.0, 0.0, 0.0 }
\definecolor{sh_keyword}{rgb}{0.0, 0.0, 0.0 }
\definecolor{sh_string}{rgb}{0.0, 0.0, 0.0 }

\lstset{ %
	language=Java,
	basicstyle=\small\ttfamily,
	fontadjust,
	xrightmargin=1mm,
	xleftmargin=5mm,
	tabsize=2,
	columns=flexible,
	showstringspaces=false,
	rulesepcolor=\color{black},
	showspaces=false,showtabs=false,tabsize=2,
	stringstyle=\color{sh_string},
	keywordstyle=\color{sh_keyword}\bfseries,
	commentstyle=\color{sh_comment}\itshape,
	captionpos=t,
	lineskip=-0.3em
}

%\makeatletter
%\def\l@lstlisting#1#2{\@dottedtocline{1}{0em}{1.5em}{\lstlistingname\space{#1}}{#2}}
%\makeatother

% Anhangsverzeichnis
\usepackage[nohints]{minitoc} %Anhangsverzeichnis

\makeatletter
\newcounter{fktnr}\setcounter{fktnr}{0}
\newcounter{subfktnr}[fktnr]\setcounter{subfktnr}{0}

\renewcommand\thesubfktnr{\arabic{fktnr}.\arabic{subfktnr}}
\newcounter{anhangcounter}
\newcommand{\blatt}{\stepcounter{anhangcounter}}

\newcommand{\anhang}[1]{\setcounter{anhangcounter}{0}\refstepcounter{fktnr}
	\addcontentsline{fk}{subsection}{Anhang~\thefktnr: \hspace*{1em}#1}
	\subsection*{{Anhang~\thefktnr{} \hspace*{1em} #1 \hspace*{-1em}}}
}

\newcommand{\subanhang}[1]{\setcounter{anhangcounter}{0}\refstepcounter{subfktnr}
	\addcontentsline{fk}{subsubsection}{Anhang~\thesubfktnr: \hspace*{1em}#1}
	\subsubsection*{{Anhang~\thesubfktnr{} \hspace*{1em} #1 \hspace*{-1em}}}
}

\newcommand{\anhangsverzeichnis}{\mtcaddsection{\subsection*{Anhangsverzeichnis \@mkboth{FKT}{FKT}}}\@starttoc{fk}\newpage}

% Abkürzungsverzeichnis
\usepackage[acronym,  % create list of acronyms
	nonumberlist,
	toc,
	section,
	nomain,           % don't need main glossary for this example
	hyperfirst=false  % don't hyperlink first use
]{glossaries}
\newglossarystyle{mylist}{%
	\setglossarystyle{long}% base this style on the list style
	\renewcommand*{\glossaryentryfield}[5]{%
		\glsentryitem{##1}\textbf{##2} & ##3 \\}%
}

\newacronym{EX}{EX}{Example}
\makeglossaries\makeglossaries{}